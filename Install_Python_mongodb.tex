\documentclass[10pt,a4paper]{article}

%%%%%%%%%%%%%%%%% CJK 中文版面控制  %%%%%%%%%%%%%%%%%%%%%%%%%%%%%%
%\usepackage{CJK} % CTEX-CJK 中文支持                            %
\usepackage{xeCJK} % seperate the english and chinese		 %
%\usepackage{CJKutf8} % Texlive 中文支持                          %
\usepackage{CJKnumb} %中文序号                                   %
\usepackage{indentfirst} % 中文段落首行缩进                      %
%\setlength\parindent{22pt}       % 段落起始缩进量               %
\renewcommand{\baselinestretch}{1.2} % 中文行间距调整            %
\setlength{\textwidth}{16cm}                                     %
\setlength{\textheight}{24cm}                                    %
\setlength{\topmargin}{-1cm}                                     %
\setlength{\oddsidemargin}{0.1cm}                                %
\setlength{\evensidemargin}{\oddsidemargin}                      %
%%%%%%%%%%%%%%%%%%%%%%%%%%%%%%%%%%%%%%%%%%%%%%%%%%%%%%%%%%%%%%%%%%

\usepackage{amsmath,amsthm,amsfonts,amssymb,bm}          %数学公式
\usepackage{mathrsfs}                                    %英文花体

\usepackage{xcolor}                                        %使用默认允许使用颜色

\usepackage{fontspec} % use to set font
\setCJKmainfont{SimSun}
\XeTeXlinebreaklocale "zh"  % Auto linebreak for chinese
\XeTeXlinebreakskip = 0pt plus 1pt % Auto linebreak for chinese

\usepackage{longtable}                                   %使用长表格

%%%%%%%%%%%%%%%%%%%%%%%%%  参考文献引用 %%%%%%%%%%%%%%%%%%%%%%%%%%%
%%尽量使用 BibTeX(含有超链接,数据库的条目URL即可)                %
%%%%%%%%%%%%%%%%%%%%%%%%%%%%%%%%%%%%%%%%%%%%%%%%%%%%%%%%%%%%%%%%%%%

\usepackage[numbers,sort&compress]{natbib} %紧密排列             %
\usepackage[sectionbib]{chapterbib}        %每章节单独参考文献   %
\usepackage{hypernat}                                                                         %
\usepackage[bookmarksopen=true,pdfstartview=FitH,CJKbookmarks]{hyperref}              %
\hypersetup{bookmarksnumbered,colorlinks,linkcolor=green,citecolor=blue,urlcolor=red}         %
%参考文献含有超链接引用时需要下列宏包,注意与natbib有冲突        %
%\usepackage[dvipdfm]{hyperref}                                  %
%\usepackage{hypernat}                                           %
\newcommand{\upcite}[1]{\hspace{0ex}\textsuperscript{\cite{#1}}} %

%%%%%%%%%%%%%%%%%%%%%%%%%%%%%%%%%%%%%%%%%%%%%%%%%%%%%%%%%%%%%%%%%%%%%%%%%%%%%%%%%%%%%%%%%%%%%%%
%\AtBeginDvi{\special{pdf:tounicode GBK-EUC-UCS2}} %CTEX用dvipdfmx的话,用该命令可以解决      %
%						   %pdf书签的中文乱码问题		      %
%%%%%%%%%%%%%%%%%%%%%%%%%%%%%%%%%%%%%%%%%%%%%%%%%%%%%%%%%%%%%%%%%%%%%%%%%%%%%%%%%%%%%%%%%%%%%%%

%%%%%%%%%%%%%%%%%%%%%  % EPS 图片支持  %%%%%%%%%%%%%%%%%%%%%%%%%%%
\usepackage{graphicx}                                            %
%%%%%%%%%%%%%%%%%%%%%%%%%%%%%%%%%%%%%%%%%%%%%%%%%%%%%%%%%%%%%%%%%%


\begin{document}
%\CJKindent     %在CJK环境中,中文段落起始缩进2个中文字符
\graphicspath{{figure/}}
%
\renewcommand{\abstractname}{\small{\CJKfamily{hei} 摘\quad 要}} %\CJKfamily{hei} 设置中文字体,字号用\big \small来设
\renewcommand{\refname}{\centering\CJKfamily{hei} 参考文献}
%\renewcommand{\figurename}{\CJKfamily{hei} 图.}
\renewcommand{\figurename}{{\bf Fig}.}
%\renewcommand{\tablename}{\CJKfamily{hei} 表.}
\renewcommand{\tablename}{{\bf Tab}.}

%将图表的Caption写成 图(表) Num. 格式
\makeatletter
\long\def\@makecaption#1#2{%
  \vskip\abovecaptionskip
  \sbox\@tempboxa{#1. #2}%
  \ifdim \wd\@tempboxa >\hsize
    #1. #2\par
  \else
    \global \@minipagefalse
    \hb@xt@\hsize{\hfil\box\@tempboxa\hfil}%
  \fi
  \vskip\belowcaptionskip}
\makeatother

\newcommand{\keywords}[1]{{\hspace{0\ccwd}\small{\CJKfamily{hei} 关键词:}{\hspace{2ex}{#1}}\bigskip}}

%%%%%%%%%%%%%%%%%%中文字体设置%%%%%%%%%%%%%%%%%%%%%%%%%%%
%默认字体 defalut fonts \TeX 是一种排版工具 \\		%
%{\bfseries 粗体 bold \TeX 是一种排版工具} \\		%
%{\CJKfamily{song}宋体 songti \TeX 是一种排版工具} \\	%
%{\CJKfamily{hei} 黑体 heiti \TeX 是一种排版工具} \\	%
%{\CJKfamily{kai} 楷书 kaishu \TeX 是一种排版工具} \\	%
%{\CJKfamily{fs} 仿宋 fangsong \TeX 是一种排版工具} \\	%
%%%%%%%%%%%%%%%%%%%%%%%%%%%%%%%%%%%%%%%%%%%%%%%%%%%%%%%%%

%\addcontentsline{toc}{section}{Bibliography}

%-------------------------------The Title of The Paper-----------------------------------------%
\title{Python及Pymatgen软件的安装}
%----------------------------------------------------------------------------------------------%

%----------------------The Authors and the address of The Paper--------------------------------%
\author{
\small
%Author1, Author2, Author3\footnote{Communication author's E-mail} \\    %Authors' Names	       %
\small
%(The Address,City Post code)						%Address	       %
}
\date{}					%if necessary					       %
%----------------------------------------------------------------------------------------------%
\maketitle

%-------------------------------------------------------------------------------The Abstract and the keywords of The Paper----------------------------------------------------------------------------%
%\begin{abstract}
%The content of the abstract
%\end{abstract}

%\keywords {Keyword1; Keyword2; Keyword3}
%-----------------------------------------------------------------------------------------------------------------------------------------------------------------------------------------------------%

%----------------------------------------------------------------------------------------The Body Of The Paper----------------------------------------------------------------------------------------%
%Introduction
\textrm{Pymatgen}的安装需要\textrm{Python}2.7以上版本,需要调用\textrm{Tck/Tk}(要求版本8.6以上),因此手动安装时配置如下:

\textbf{Tcl/Tk的安装配置}\\
./configure -\/-prefix=/share/home/jiangjun/Softs/tcl8.6.5 -\/-enable-shared -\/-enable-64bit -\/-enable-symbols\\
make \&\& make install\\
./configure -\/-prefix=/share/home/jiangjun/Softs/tk8.6.5 -\/-enable-shared -\/-enable-64bit -\/-enable-symbols\\
make \&\& make install

\textcolor{blue}{\textrm{lapack}和\textrm{atlas}的安装参见文档}。\textcolor{red}{\textbf{建议:用OpenBlas代替atlas更方便}}\\
安装\textrm{OpenBlas}非常简单:\;\; \textrm{make CC=gcc FC=gfortran}\\

\textcolor{red}{请注意}:为了让\textrm{lapack}和\textrm{atlas}最后生成动态库函数,编译\textrm{lapack}库时,在\textrm{make.inc}中\textcolor{red}{每个编译选项都加上-fPIC -m64}\\
如\\
FORTRAN  = gfortran \\
OPTS     = -O3 -std=legacy -m64 -fno-second-underscore -fPIC -c\\
DRVOPTS  = \$(OPTS)\\
NOOPT    = -O0 -frecursive -fPIC -m64\\
LOADER   = gfortran \\
LOADOPTS = -fPIC -m64\\
\\
CC = gcc\\
CFLAGS = -O3 -fPIC -m64\\

在非\textrm{root}权限下,\textrm{atlas}推荐的编译选项:\\

../configure -b 64 -C ic gcc -C if gfortran -Fa alg -fPIC -\/-prefix=指定安装目录 -\/-with-netlib-lapack=静态库函数liblapack.a的绝对路径\\

手工将静态库编译为动态库\\
gfortran -fPIC -m64 -shared liblapack.a -o liblapack.so\\
gfortran -fPIC -m64 -shared libblas.a -o libblas.so\\


参见文档\textcolor{red}{\textrm{ATLAS\_Numpy\_Scipy\_Theano}的环境搭建.pdf}

\fbox{
	\parbox{\textwidth}{
\textrm{python build finished successfully!\\
The necessary bits to build these optional modules were not found:\\
\_bz2                  \_curses               \_curses\_panel      \\
\_dbm                  \_gdbm                 \_hashlib           \\
\_lzma                 \_sqlite3              \_ssl               \\
\_tkinter              readline                                 \\
To find the necessary bits, look in setup.py in detect\_modules() for the module's name.\\
\\
\\
The following modules found by detect\_modules() in setup.py, have been\\
built by the Makefile instead, as configured by the Setup files:\\
\_abc                  atexit                pwd                \\
time                                                           \\
\\
\\
Failed to build these modules:\\
\_ctypes                         \\                               
\\
\\
Could not build the ssl module!\\
Python requires an OpenSSL 1.0.2 or 1.1 compatible libssl with X509\_VERIFY\_PARAM\_set1\_host().\\
LibreSSL 2.6.4 and earlier do not provide the necessary APIs, https://github.com/libressl-portable/portable/issues/381}}}

需要安装的库函数:\\
\fbox{
	\parbox{\textwidth}{
		\textrm{
yum -y groupinstall "Development tools" \\
yum -y install zlib-devel bzip2-devel openssl-devel ncurses-devel sqlite-devel readline-devel tk-devel gdbm-devel db4-devel libpcap-devel xz-devel \\
yum install -y libffi-devel zlib1g-dev \\
yum install zlib* -y}}}

\textbf{python的源码包安装配置}\\
./configure CC=gcc -\/-prefix=/share/home/jiangjun/Softs/Python-2.7.11  -\/-enable-universalsdk -\/-with-universal-archs=``64-bit''  -\/-with-cxx-main=g++ -\/-with-tcltk-includes='-I/share/home/jiangjun/lib/tcltk/include' -\/-with-tcltk-libs='/share/home/jiangjun/lib/tcltk/lib/libtcl8.6.so /share/home/jiangjun/lib/tcltk/lib/libtk8.6.so'\\
make \&\& make install

\textbf{类似地,python3的源码包安装配置}\\
./configure CC=gcc -\/-prefix=/share/home/jiangjun/Softs/Python-3.9.5  -\/-enable-universalsdk -\/-with-universal-archs=``64-bit''  -\/-with-cxx-main=g++ -\/-with-tcltk-includes='-I/share/home/jiangjun/lib/tcltk/include' -\/-with-tcltk-libs='/share/home/jiangjun/lib/tcltk/lib/libtcl8.6.so /share/home/jiangjun/lib/tcltk/lib/libtk8.6.so' -\/-LDFLAGS='-L/usr/local/lib" CPPFLAGS="-I/usr/local/include' -\/-with-openssl=/home/TEST/.local/openssl \\
make \&\& make install

\textbf{setuptools安装} (参见文档 \textcolor{blue}{\textrm{Python}的安装过程(含\textrm{setuptools}).pdf})\\
python setup.py build\\
python setup.py install


采用\textrm{setuptools},在可以实现在普通用户账号下安装模块(\textrm{ez\_setup.py}是\textrm{python}官方给出的一个安装\textrm{setuptools}的工具),如:\\
/share/software/python-2.7.10/bin/python ez\_setup.py -\/-user jiangjun\\
\textrm{easy\_install}会被安装在$\sim$/.local/bin/ 目录下)\\\\

在此基础上,通过选项 \textcolor{red}{-d} 可将各种模块安装到普通用户账号下的指定目录:\\
$\sim$/.local/bin/easy\_install \textcolor{red}{-d} $\sim$/.local/lib/python2.7/site-package \textcolor{blue}{\underline{安装模块}}

在$\sim$/.bashrc 中加入下列变量(即\textrm{PYTHONPATH}包含新增模块目录):\\
\textcolor{red}{export PYTHONPATH=\$PATHONPATH:$\sim$/.local/lib/python2.7/site-package}\\

有此\textcolor{magenta}{PYTHONPATH}也可以用命令行,将有关模块直接安装到指定环境下\\
\textcolor{red}{python setup.py install -\/-prefix=$\sim$/.local}

参见文档\\
\textcolor{blue}{\textrm{Python}环境变量\textrm{PYTHONPATH}设置和\textrm{easy\_install}简单使用.pdf}

可以在配置文件 \textcolor{blue}{$\sim$/.pydistutils.cfg} 中指定下载镜像:\\
$[$easy\_install$]$\\
\# index\_url = \url{http://pypi.douban.com/simple/} \\
\# index\_url = \url{http://e.pypi.python.org/simple} \\

通用\textrm{pip}安装模块命令:\\
\textcolor{blue}{\textrm{pip install Module==version -i http://mirrors.aliyun.com/pypi/simple -\/-trusted-host mirrors.aliyun.com -\/-user}}\\
\begin{itemize}
	\item 模块用\textrm{==}指定版本
	\item \textrm{-i}和\textrm{-\/-trusted-host}指定\textrm{pip}下载库位置
	\item \textrm{-\/-}user指定在当前用户的{$\sim$/.local/lib/python}版本下安装模块
\end{itemize}

如果采用\textrm{pip}安装模块,建议通过配置文件 \textcolor{blue}{$\sim$/.pip/pip.conf} 中指定有关参数:\\
$[$global$]$\\
timeout = 6000\\
index-url = \url{http://pypi.douban.com/simple/} \\
$[$install$]$\\
use-mirrors = true\\
mirrors = \url{http://pypi.douban.com/simple/} \\
trusted-host = \url{pypi.douban.com} \\
\underline{install-option}=-\/-prefix=``$\sim$/.local '' \# \textcolor{red}{指定安装路径选项,而且引号不能少} \\

指定\textrm{pip3}的安装源(以阿里云为例)
\begin{itemize}
	\item \textrm{pip3~config~set~global.index-url~\url{http://mirrors.aliyun.com/pypi/simple/}}
	\item \textrm{pip3~config~set~install.trusted-host~mirrors.aliyun.com}
	\item \textrm{pip3~config~list}
	\item \textrm{pip3~config~set~global.index-url~\url{http://mirrors.aliyun.com/pypi/simple/}}
	\item \textrm{pip3~config~set~install.trusted-host mirrors.aliyun.com}
\end{itemize}

\textcolor{violet}{在无法联网的服务器上用\textrm{pip}安装\textrm{Python}模块的一些处理:}\\
下载安装模块:~ \textrm{\url{https://pypi.python.org/simple/}}模块.tar.gz\:\:\:(\textcolor{red}{无需解压})\\
\textcolor{purple}{pip install 模块.版本.tar.gz}\\


\textrm{Python}运行\textrm{pip}时,如果提示\textcolor{red}{\underline{\textrm{Can't connect to HTTPS URL because the SSL module is not available.-skipping}}},就需要安装\textrm{openssl}(1.1或1.02版)或\textrm{libressl}的2.6.4版本:
\begin{itemize}
	\item \textrm{wget~\url{http://www.openssl.org/source/openssl-xxx.tar.gz}}
	\item \textrm{tar~xvzf~openssl-xxx.tar.gz}
	\item \textrm{./config~~-prefix=/usr/local/openssl-xxx~~--openssldir=/usr/local/openssl-xxx/openssl~~no-zlib} ~~~\#建议加上\textrm{no-zlib}~否则会出现\textrm{undefined symbol:~SSL\_CTX\_get0\_param}错误
	\item \textrm{make~\&\&~make~install}
	\item \textrm{echo~``/usr/local/openssl-xxx/lib''~>>~/etc/ld.so.conf}
	\item \textrm{ldconfig -v}
\end{itemize}
重新编译和安装\textrm{Python},用选项\textrm{-\/-with-openssl=\$openssl~}的安装目录,在\textrm{Ubuntu}下记得修改\textrm{Module/Setup}文件:~
\begin{itemize}
	\item \textrm{vim~/root/Python-XXXX/Modules/Setup}
	\item 修改结果如下:\\
		\fbox{
			\parbox{0.9\textwidth}{
   \textrm{\#~Socket module helper for socket(2)\\ 
    \_socket socketmodule.c timemodule.c\\ 
    \#~Socket module helper for SSL support; you must comment out the other\\
    \#~socket line above, and possibly edit the SSL variable: \\
    SSL=/usr/local/openssl-xxx \\
    \_ssl~\_ssl.c \ \\
    -DUSE\_SSL~~-I\$(SSL)/include ~~-I\$(SSL)/include/openssl \\ 
    -L\$(SSL)/lib -lssl -lcrypto }}}
    \item 运行\textrm{python}
    \item \textrm{import~~ssl}测试正常即可
\end{itemize}
\textcolor{red}{如果编译中使用动态库\textrm{libssl}有问题,建议使用静态库编译}

指定\textrm{pip3}的安装源(以阿里云为例)\\
\begin{itemize}
	\item \textrm{pip3~config~set~global.index-url~\url{http://mirror.aliyun.com/pypi/simple/}}
	\item \textrm{pip3~config~set~install.trusted-host~mirror.aliyun.com}
	\item \textrm{pip3~config~list}
	\item \textrm{pip3~config~set~global.index-url~\url{http://mirror.aliyun.com/pypi/simple/}}
	\item \textrm{pip3~config~set~install.trusted-host~mirror.aliyun.com}
	\item \textrm{pip3~config~list}
\end{itemize}

在此基础上,准备手动安装\textrm{numpy}、\textrm{scipy}等模块,建议指定环境变量\textrm{ATLAS}(版本3.8.4)、\textrm{LAPACK}(版本3.6.0)、\textrm{BLAS}\\ 
\textcolor{red}{注意:环境变量要具体到绝对路径(包括文件名)},如:\\
\textcolor{blue}{export ATLAS=/share/home/jiangjun/lib/atlas3.8.4/lib/libatlas.so}\\
\textcolor{blue}{export LAPACK=/share/home/jiangjun/lib/atlas3.8.4/lib/liblapack.so}\\
\textcolor{blue}{export BLAS=/share/home/jiangjun/lib/atlas3.8.4/lib/libblas.so}\\
分别进入\textrm{numpy}和\textrm{scipy}目录\\ 
在文件 site.cfg 中指定有关库函数的参数(\textcolor{red}{注意:写绝对路径})\\
\textcolor{green}{$[$ALL$]$}\\
library\_dirs =/share/home/jiangjun/lib/atlas3.8.4/lib \\
include\_dirs =/share/home/jiangjun/lib/atlas3.8.4/include \\
src\_dir=/share/home/jiangjun/lib/atlas3.8 \\
search\_static\_first=0 \\
\#
\\
\# Atlas\\
\# -\/-\/-\/-\/-\\
\# Atlas is an open source optimized implementation of the BLAS and Lapack\\
\# routines. Numpy will try to build against Atlas by default when available in\\
\# the system library dirs. To build numpy against a custom installation of\\
\# Atlas you can add an explicit section such as the following. Here we assume\\
\# that Atlas was configured with ``prefix=/opt/atlas``.\\
\#\\
\textcolor{green}{$[$atlas$]$}\\
library\_dirs = /share/home/jiangjun/lib/atlas3.8.4/lib \\
include\_dirs = /share/home/jiangjun/lib/atlas3.8.4/include \\
atlas\_libs=lapack, f77blas, cblas, atlas \\

\end{itemize}<++>

在此基础上,准备手动安装\textrm{numpy}、\textrm{scipy}等模块,建议指定环境变量\textrm{ATLAS}(版本3.8.4)、\textrm{LAPACK}(版本3.6.0)、\textrm{BLAS}\\ 
\textcolor{red}{注意:环境变量要具体到绝对路径(包括文件名)},如:\\
\textcolor{blue}{export ATLAS=/share/home/jiangjun/lib/atlas3.8.4/lib/libatlas.so}\\
\textcolor{blue}{export LAPACK=/share/home/jiangjun/lib/atlas3.8.4/lib/liblapack.so}\\
\textcolor{blue}{export BLAS=/share/home/jiangjun/lib/atlas3.8.4/lib/libblas.so}\\
分别进入\textrm{numpy}和\textrm{scipy}目录\\ 
在文件 site.cfg 中指定有关库函数的参数(\textcolor{red}{注意:写绝对路径})\\
\textcolor{green}{$[$ALL$]$}\\
library\_dirs =/share/home/jiangjun/lib/atlas3.8.4/lib \\
include\_dirs =/share/home/jiangjun/lib/atlas3.8.4/include \\
src\_dir=/share/home/jiangjun/lib/atlas3.8 \\
search\_static\_first=0 \\
\#
\\
\# Atlas\\
\# -\/-\/-\/-\/-\\
\# Atlas is an open source optimized implementation of the BLAS and Lapack\\
\# routines. Numpy will try to build against Atlas by default when available in\\
\# the system library dirs. To build numpy against a custom installation of\\
\# Atlas you can add an explicit section such as the following. Here we assume\\
\# that Atlas was configured with ``prefix=/opt/atlas``.\\
\#\\
\textcolor{green}{$[$atlas$]$}\\
library\_dirs = /share/home/jiangjun/lib/atlas3.8.4/lib \\
include\_dirs = /share/home/jiangjun/lib/atlas3.8.4/include \\
atlas\_libs=lapack, f77blas, cblas, atlas \\
\\
\\
\textcolor{green}{$[$blas\_opt$]$}\\
library\_dirs = /share/home/jiangjun/lib/atlas3.8.4/lib \\
include\_dirs = /share/home/jiangjun/lib/atlas3.8.4/include \\
blas\_libs=f77blas, cblas, atlas \\
\\
\textcolor{green}{$[$lapack\_opt$]$}\\
library\_dirs = /share/home/jiangjun/lib/atlas3.8.4/lib \\
include\_dirs = /share/home/jiangjun/lib/atlas3.8.4/include \\
lapack\_libs=lapack, atlas \\
\\
\textcolor{green}{$[$amd$]$}\\
amd\_libs = amd \\
\#\\
\textcolor{green}{$[$umfpack$]$} \\
umfpack\_libs = umfpack \\
\\
\textbf{用Intel\_mkl库支持numpy和scipy}\\
\textcolor{green}{$[$mkl$]$}\\
library\_dirs = \$intel编译器安装目录/mkl/lib/intel64/ \\
include\_dirs = \$intel编译器安装目录/mkl/include \\
mkl\_libs = \textcolor{red}{mkl\_rt~ mkl\_blas95\_lp64 mkl\_core~ mkl\_intel\_lp64~ mkl\_intel\_thread}\\
lapack\_libs =\textcolor{red}{mkl\_lapack95\_lp64}\\

\textcolor{blue}{修改\$numpy-目录/numpy/distutils/intelccompiler.py}:\\
将\textcolor{magenta}{self.cc\_exe(class intel或class intelem里的)}为:\\
self.cc\_exe = 'icc -O3 -g -fPIC -fp-model strict -fomit-frame-pointer -openmp -xhost'\\
numpy编译:\\
\textcolor{blue}{python setup.py config -\/-compiler=intel build\_clib -\/-compiler=intel build\_ext -\/-compiler=intel install}\\
64位将\textcolor{blue}{intel}改为\textcolor{red}{intelem}\\
scipy编译:\\
\textcolor{blue}{python setup.py config -\/-compiler=intel -\/-fcompiler=intel build\_clib -\/-compiler=intel -\/-fcompiler=intel build\_ext -\/-compiler=intel -\/-fcompiler=intel install}\\
64位同numpy修改命令行\\



python setup.py build\\
sudo python setup.py install\\
\\
在\textrm{Ubuntu}系统中,如果有超级用户权限,则可以简单处理为:\\
sudo apt-get install libopenblas-dev liblapack-dev\\
export BLAS=/usr/lib/libblas.so\\
export LAPACK=/usr/lib/liblapack.so\\
\\
在此基础上完成安装:\\
pip install numpy\\
pip install scipy\\

\subsection{关于\rm{python}的图形界面\rm{GUI}}
直接用\textrm{pip/pip3}安装\\
\textrm{pip3~install~-i~\url{https://pypi.tuna.tsinghua.edu.cn/simple}~wxpython}\\
会有问题
解决办法是到网页\\
\url{https://extras.wxpython.org/wxPython4/extras/linux/gtk3/}\\
直接选择\textcolor{red}{合适的\textrm{whl}版本},用命令\\
如:~\textrm{pip3~install~ ~/Downloads/wxPython-4.2.0-cp38-cp38-linux_x86_64.whl}\\
安装

\textbf{virtualenv}:~用来建立一个虚拟的\textrm{python}环境,一个专属于项目的\textrm{python}环境。
用\textrm{virtualenv}来保持一个干净的环境非常有用
\begin{enumerate}
	\item 基本使用: 通过\textrm{pip}安装\textrm{virtualenv}:\\
		\textrm{pip~install~virtualenv}
	\item 测试安装:\\
		\textrm{virtualenv~-\/-version}
	\item 为一个工程项目搭建一个虚拟环境:\\
		\textrm{cd~my\_project}\\
		\textrm{virtualenv~my\_project_env}
	\item 如果存在多个\textrm{python}解释器,可以选择指定一个\textrm{Python}解释器(比如``\textrm{python2.7}``),没有指定则由系统默认的解释器来搭建:\\
		\textrm{virtualenv~-p~/usr/bin/python2.7~my\_project\_env}\\
		将会在当前的目录中创建一个名为\textrm{my\_project\_env}的文件夹,这是一个独立的\textrm{python}运行环境,包含了\textrm{Python}可执行文件, 以及~\textrm{pip}~ 库的一份拷贝,这样就能安装其他包了,不过已经安装到系统\textrm{Python}环境中的所有第三方包都不会复制过来,这样就得到了一个不带任何第三方包的“干净”的\textrm{Python}运行环境来。
	\item 要开始使用虚拟环境,其需要被激活:\\
		\textrm{source~my\_project\_env/bin/activate}
	\item 停用虚拟环境:\\
		\textrm{deactivate}\\
		停用后将回到系统默认的\textrm{Python}解释器
\end{enumerate}

\section{关于\textrm{MongoDB}的启动}
\textrm{MongoDB}可以通过控制文件(如文件名为\textrm{mongodb.config})来控制启动,\textrm{mongodb.config}中指定各种参数:\\
\fbox{
	\parbox{\textwidth}{
\textrm{fork = true\\
dbpath = /home/jun\_jiang/WORKS/TEST/TEST\_mongo/Data\_Base\\
logpath = /home/jun\_jiang/WORKS/TEST/TEST\_mongo/mongo.log\\
logappend =true\\
journal = true\\
\# repair = true\\
bind\_ip = 127.0.0.1,192.168.113.42\#这里的\textrm{IP(192.168.113.42)}为数据库服务器的内部\textrm{IP}(即\textrm{ifconfig}命令查询到的本机\textrm{IP})\\
port = 27017}}}

如果需要在同一地点启动多个数据库,只需要修改\textrm{dbpath}即可\\

如\textrm{mongodb\_atomate.config}:\\
\fbox{
	\parbox{\textwidth}{
\textrm{fork = true\\
dbpath = /home/jun\_jiang/WORKS/TEST/TEST\_mongo/Data\_Base\_Atomate\\
logpath = /home/jun\_jiang/WORKS/TEST/TEST\_mongo/mongo.log\\
logappend =true\\
journal = true\\
\# repair = true\\
bind\_ip = 127.0.0.1,192.168.113.42\#这里的\textrm{IP(192.168.113.42)}为数据库服务器的内部\textrm{IP}(即\textrm{ifconfig}命令查询到的本机\textrm{IP})\\
port = 27017}}}

如\textrm{mongodb\_firework.config}:\\
\fbox{
	\parbox{\textwidth}{
\textrm{fork = true\\
dbpath = /home/jun\_jiang/WORKS/TEST/TEST\_mongo/Data\_Base\_Firework\\
logpath = /home/jun\_jiang/WORKS/TEST/TEST\_mongo/mongo.log\\
logappend =true\\
journal = true\\
\# repair = true\\
bind\_ip = 127.0.0.1,192.168.113.42\#这里的\textrm{IP(192.168.113.42)}为数据库服务器的内部\textrm{IP}(即\textrm{ifconfig}命令查询到的本机\textrm{IP})\\
port = 27017}}}

启动数据库的命令为:~\textrm{\textcolor{red}{mongod~-f~mongodb\_XXXX.config}}\\

进入数据库的命令为:~\textrm{\textcolor{red}{mongo}},然后用\textrm{help}检查\textrm{MongoDB}的各种命令

\section{\rm{Pymatgen}}
采用\textrm{Pymatgen}生成\textrm{VASP}~的\textrm{POTCAR}库函数时,用的命令是\\
\textcolor{red}{\textrm{pmg~config~-p~Dir-source~Dir-object}}\\
然后用\\
\textrm{pmg~config~add~PMG\_VASP\_PSP\_DIR ~Dir-object} \\
生效即可

\textrm{pmg~potcar}命令是用于根据\textrm{POTCAR}库产生对应元素的\textrm{POTCAR},在\textrm{ATOMATE}系统中,\textrm{~DIR-object}必须是\[\textrm{POT\_GGA\_PAW\_PBE},\textrm{POT\_LDA\_PAW}\]

\section{\rm{ATOMATE}下的数据库配置}
启动支持计算进程的\textrm{FireWorks}数据库:\\
\textrm{mongod~-f~./mongodb\_fireworks.config}

启动支持计算数据存储的\textrm{Atomate}数据库:\\
\textrm{mongod~-f~./mongodb\_atomate.config.config}

\begin{itemize}
	\item \textrm{ATOMATE}环境下,加载\textrm{FireWorks}的总体配置:\textrm{~/.bashrc}中写入环境变量:\\
		\textrm{export FW\_CONFIG\_FILE=<<INSTALL\_DIR>>/config/FW\_config.yaml}\\
		\textrm{FW\_config.yaml}的内容:\\
\fbox{
	\parbox{\textwidth}{
		\textrm{LAUNCHPAD\_LOC:~/home/TEST/WORKS/FireWorks\_TEST/my\_launchpad.yaml}\\
		\textrm{FWORKER\_LOC:~/home/TEST/WORKS/FireWorks\_TEST/my\_fworker.yaml} \\
		\textrm{QUEUEADAPTER\_LOC:~/home/TEST/WORKS/FireWorks\_TEST/my\_qadapter.yaml} \\
		\textrm{CONFIG\_FILE\_DIR:~/home/TEST/WORKS/FireWorks\_TEST}}}
	\item  文件\textrm{my\_launchpad.yam}~指定了\textrm{FireWorks}支持的计算流程存储的数据库的(相关登录)信息,~\textcolor{red}{(这个文件指定的参数一般不要改动)}\\
\fbox{
	\parbox{\textwidth}{
		\textrm{authsource:~fireworks}~\#指定支持计算流程的数据库为\textrm{FireWorks}\\
		\textrm{host:~localhost}~\#指定\textrm{IP}或域名\\
		\textrm{logdir:~null}\\
		\textrm{mongoclient\_kwargs:~\{\}}\\
		\textrm{name:~fireworks}~\#指定数据库登录账号为~\textrm{FireWorks}\\
		\textrm{password:~null} ~\#指定数据库登录密码\\
		\textrm{port:~27017} ~\#指定数据库端口\\
		\textrm{ssl:~false} ~\#默认为\textrm{false},以下全部空;\textrm{true},以下填入合适的参数\\
		\textrm{ssl\_ca\_certs:~null}\\
		\textrm{ssl\_certfile:~null}\\
		\textrm{ssl\_keyfile:~null}\\
		\textrm{ssl\_pem\_passphrase:~null}\\
		\textrm{strm\_lvl:~INFO}\\
		\textrm{uri\_mode:~false}\\
		\textrm{user\_indices:~[~]}\\
		\textrm{username:~null}\\
		\textrm{wf\_user\_indices:~[~]}}}
	\item 文件\textrm{my\_fworker.yaml}~指定了运行所需计算软件、计算数据存储数据位置信息\\
\fbox{
	\parbox{\textwidth}{
		\textrm{name:~mgo-test}~\#指定计算对象(任务)的名称\\
		\textrm{category:~''}\\
		\textrm{query:~'\{\}'}\\
		\textrm{env:}\\
		\hspace*{25pt}\textrm{db\_file:~/home/TEST/WORKS/Mongo\_TEST/Cal\_Data\_Base/db.json}~\#指定计算数据存储用数据库的位置,\textrm{db.json}文件包含计算存储数据库的信息\\
		\hspace*{25pt}\textrm{vasp\_cmd:~mpirun~-np~4~/home/TEST/Softwares/vasp.5.4.4/bin/vasp\_std}~\#指定核心计算所需的引擎\\
		\hspace*{25pt}\textrm{scratch\_dir:~null}}}
	\item 文件\textrm{my\_qadapter.yaml}~\# 这个文件指定了作业管理系统支持下提交计算作业的参数和形式,~\textcolor{blue}{这个文件是支持\textrm{qlaunch~rapidfire~\[-m~3\]}}命令的运行,类似于直接启动\\
		\textrm{rlaunch~-l~/home/TEST/WORKS/FireWorks\_TEST/my\_launchpad.yaml~-w~/home/TEST/WORKS/FireWorks\_TEST/my\_fworker.yaml~rapidfire}\\
\fbox{
	\parbox{\textwidth}{
		\textrm{\_fw\_name:~CommonAdapter}\\
		\textrm{\#~\_fw\_q\_type:~PBS~\#~SLURM}\\
		\textrm{\_fw\_q\_type:~LoadSharingFacility}\\
		\textrm{\#~\_fw\_template\_file:~/share/home/czjiangjun/WORKS/TEST\_ATOMATE/LSF\_template\_custom.txt}
		\textrm{rocket\_launch:~rlaunch~-w~/home/jun-jiang/WORKS/FireWorks/my\_fworker.yaml~-l~/home/jun-iang/WORKS/FireWorks/my\_launchpad.yaml~rapidfire~#~singleshot} \#~本行指定的就是每个进程命令\\ 
		\textrm{\#~rocket\_launch:~rlaunch~-c~/share/home/czjiangjun/WORKS/TEST\_FireWorks~singleshot~#rapidfire}\\
		\textrm{\#~rocket\_launch:~rlaunch~-c~/share/home/czjiangjun/WORKS/TEST\_FireWorks~singleshot}\\ 
		\textrm{\#~rocket\_launch:~bsub ~~./stand.lsf}
		\textrm{nodes:~4}\\
		\textrm{ntasks\_per\_node:~2}\\
		\textrm{\#~walltime:~24:00:00}\\
		\textrm{queue:~para2}\\
		\textrm{\#~account:~czjiangjun}\\
		\textrm{job\_name:~VASP\_Cal}\\
		\textrm{pre\_rocket:~null}\\
		\textrm{post\_rocket:~null}\\
		\# 以下是针对本单位\textrm{LSF}的改动脚本:\\
		\textrm{\#~NP\_PER\_NODE:~12 \#~(一个节点跑\textrm{n}个进程,不指定系统按照资源使用情况自主决定)}\\
		\textrm{MPI\_TYPE:~openmpi ~\#~(选择mpi的类型)}\\
		\textrm{MPI\_HOME:~/home/jun-jiang/Softswares/openmpi-1.8.4 ~\#~(\textrm{mpi}的路径)}\\
		\textrm{RUN:~RAW}\\
		\textrm{\#~OMP\_NUM\_THREADS=4 ~~\#(如果使用\textrm{OpenMP}或者是单节点启动多个进程,需要通过该参数修正资源的分配,该值表示一个进程中运行了4个线程)}\\
		\textrm{\_q\_commands\_override:}\\
		\hspace*{25pt}\textrm{submit\_cmd:~bsub}\\
		\hspace*{25pt}\#~\textrm{submit\_cmd:~./}\\
		\hspace*{25pt}\#~\textrm{status\_cmd:~bstatus}}}
\end{itemize}

计算结果数据库配置文件\textrm{db.json}\\
\fbox{
	\parbox{\textwidth}{
\{\\
	\hspace*{25pt}\textrm{"host":~"localhost",} ~\#指定计算结果数据库的\textrm{IP}或域名\\
	\hspace*{25pt}\textrm{"port":~27018,}~\#指定计算结果数据库的端口\\
	\hspace*{25pt}\textrm{"database":~"Cal\_Data",}~\#指定计算结果数据的数据库名\\
	\hspace*{25pt}\textrm{"collection":~"tasks",}~\#\textrm{FireWorks}中的\textrm{FireTasks}\\ 
	\hspace*{25pt}\textrm{"admin\_user":~"",}~\# 指定不同级别用户的账号和密码\\
	\hspace*{25pt}\textrm{"admin\_password":~"",}\\
	\hspace*{25pt}\textrm{"readonly\_user":~"", }\\
	\hspace*{25pt}\textrm{"readonly\_password":~"",}\\
	\hspace*{25pt}\textrm{"aliases":~\{\}}\\
\}}}
计算结果数据库配置文件\textrm{db.json}是关联
\begin{enumerate}
	\item 计算过程(通过文件\textcolor{blue}{\textrm{my\_fworker.yaml}}),确保\textrm{FireWorks}支持的计算流程结果流向数据库\textrm{db.json}
	\item 计算结果(通过文件\textcolor{blue}{\textrm{mgo\_analysis.py}}),\textrm{Pymatgen}从\textrm{db.json}数据库提取数据,完成可执行的结果分析
\end{enumerate}

执行任务的步骤
\begin{itemize}
	\item 启动计算进程管理数据库、启动计算数据存储数据库
	\item \textrm{lpad~reset} ~\#~清理以往进程 ~(如果是从头开始的进程)
	\item \textrm{python3~mgo\_bandstructure.py} ~\#~启动全部计算进程的设置
	\item \textrm{qlaunch~rapidfire~[~-m~3~] \#~有任务管理系统支持作业提交 ~/\\
		rlaunch~-l~/home/TEST/WORKS/FireWorks\_TEST/my\_launchpad.yaml~-w~/home/TEST/WORKS/FireWorks\_TEST/my\_fworker.yaml~rapidfire} ~\#~单机提交作业\\ 
		\#启动\textrm{FireWorks}支持的计算任务
	\item \textrm{python3~mgo\_analysis.py}~\#~计算结果数据分析
\end{itemize}

%-------------------The Figure Of The Paper------------------
%\begin{figure}[h!]
%\centering
%\includegraphics[height=3.35in,width=2.85in,viewport=0 0 400 475,clip]{PbTe_Band_SO.eps}
%\hspace{0.5in}
%\includegraphics[height=3.35in,width=2.85in,viewport=0 0 400 475,clip]{EuTe_Band_SO.eps}
%\caption{\small Band Structure of PbTe (a) and EuTe (b).}%(与文献\cite{EPJB33-47_2003}图1对比)
%\label{Pb:EuTe-Band_struct}
%\end{figure}

%-------------------The Equation Of The Paper-----------------
%\begin{equation}
%\varepsilon_1(\omega)=1+\frac2{\pi}\mathscr P\int_0^{+\infty}\frac{\omega'\varepsilon_2(\omega')}{\omega'^2-\omega^2}d\omega'
%\label{eq:magno-1}
%\end{equation}

%\begin{equation} 
%\begin{split}
%\varepsilon_2(\omega)&=\frac{e^2}{2\pi m^2\omega^2}\sum_{c,v}\int_{BZ}d{\vec k}\left|\vec e\cdot\vec M_{cv}(\vec k)\right|^2\delta [E_{cv}(\vec k)-\hbar\omega] \\
% &= \frac{e^2}{2\pi m^2\omega^2}\sum_{c,v}\int_{E_{cv}(\vec k=\hbar\omega)}\left|\vec e\cdot\vec M_{cv}(\vec k)\right|^2\dfrac{dS}{\nabla_{\vec k}E_{cv}(\vec k)}
% \end{split}
%\label{eq:magno-2}
%\end{equation}

%-------------------The Table Of The Paper----------------------
%\begin{table}[!h]
%\tabcolsep 0pt \vspace*{-12pt}
%\caption{The representative $\vec k$ points contributing to $\sigma_2^{xy}$ of interband transition in EuTe around 2.5 eV.}
%\label{Table-EuTe_Sigma}
%\begin{minipage}{\textwidth}
%%\begin{center}
%\centering
%\def\temptablewidth{1.01\textwidth}
%\rule{\temptablewidth}{1pt}
%\begin{tabular*} {\temptablewidth}{@{\extracolsep{\fill}}cccccc}

%-------------------------------------------------------------------------------------------------------------------------
%&Peak (eV)  & {$\vec k$}-point            &Band{$_v$} to Band{$_c$}  &Transition Orbital
%Components\footnote{波函数主要成分后的括号中,$5s$、$5p$和$5p$、$4f$、$5d$分别指碲和铕的原子轨道。} &Gap (eV)   \\ \hline
%-------------------------------------------------------------------------------------------------------------------------
%&2.35       &(0,0,0)         &33$\rightarrow$34    &$4f$(31.58)$5p$(38.69)$\rightarrow$$5p$      &2.142   \\% \cline{3-7}
%&       &(0,0,0)         &33$\rightarrow$34    &$4f$(31.58)$5p$(38.69)$\rightarrow$$5p$      &2.142   \\% \cline{3-7}
%-------------------------------------------------------------------------------------------------------------------------

%\end{tabular*}
%\rule{\temptablewidth}{1pt}\\
%%\end{center}
%\end{minipage}
%\end{table}

%-------------------The Long Table Of The Paper--------------------
%\begin{small}
%%\begin{minipage}{\textwidth}
%%\begin{longtable}[l]{|c|c|cc|c|c|} %[c]指定长表格对齐方式
%\begin{longtable}[c]{|c|c|p{1.9cm}p{4.6cm}|c|c|}
%\caption{Assignment for the peaks of EuB$_6$}
%\label{tab:EuB6-1}\\ %\\长表格的caption中换行不可少
%\hline
%%
%--------------------------------------------------------------------------------------------------------------------------------
%\multicolumn{2}{|c|}{\bfseries$\sigma_1(\omega)$谱峰}&\multicolumn{4}{c|}{\bfseries部分重要能带间电子跃迁\footnotemark}\\ \hline
%\endfirsthead
%--------------------------------------------------------------------------------------------------------------------------------
%%
%\multicolumn{6}{r}{\it 续表}\\
%\hline
%--------------------------------------------------------------------------------------------------------------------------------
%标记 &峰位(eV) &\multicolumn{2}{c|}{有关电子跃迁} &gap(eV)  &\multicolumn{1}{c|}{经验指认} \\ \hline
%\endhead
%--------------------------------------------------------------------------------------------------------------------------------
%%
%\multicolumn{6}{r}{\it 续下页}\\
%\endfoot
%\hline
%--------------------------------------------------------------------------------------------------------------------------------
%%
%%\hlinewd{0.5$p$t}
%\endlastfoot
%--------------------------------------------------------------------------------------------------------------------------------
%%
%% Stuff from here to \endlastfoot goes at bottom of last page.
%%
%--------------------------------------------------------------------------------------------------------------------------------
%标记 &峰位(eV)\footnotetext{见正文说明。} &\multicolumn{2}{c|}{有关电子跃迁\footnotemark} &gap(eV) &\multicolumn{1}{c|}{经验指认\upcite{PRB46-12196_1992}}\\ \hline

%-------------------The Figure Of The Paper------------------
%\begin{figure}[h!]
%\centering
%\includegraphics[height=3.35in,width=2.85in,viewport=0 0 400 475,clip]{PbTe_Band_SO.eps}
%\hspace{0.5in}
%\includegraphics[height=3.35in,width=2.85in,viewport=0 0 400 475,clip]{EuTe_Band_SO.eps}
%\caption{\small Band Structure of PbTe (a) and EuTe (b).}%(与文献\cite{EPJB33-47_2003}图1对比)
%\label{Pb:EuTe-Band_struct}
%\end{figure}

%-------------------The Equation Of The Paper-----------------
%\begin{equation}
%\varepsilon_1(\omega)=1+\frac2{\pi}\mathscr P\int_0^{+\infty}\frac{\omega'\varepsilon_2(\omega')}{\omega'^2-\omega^2}d\omega'
%\label{eq:magno-1}
%\end{equation}

%\begin{equation} 
%\begin{split}
%\varepsilon_2(\omega)&=\frac{e^2}{2\pi m^2\omega^2}\sum_{c,v}\int_{BZ}d{\vec k}\left|\vec e\cdot\vec M_{cv}(\vec k)\right|^2\delta [E_{cv}(\vec k)-\hbar\omega] \\
% &= \frac{e^2}{2\pi m^2\omega^2}\sum_{c,v}\int_{E_{cv}(\vec k=\hbar\omega)}\left|\vec e\cdot\vec M_{cv}(\vec k)\right|^2\dfrac{dS}{\nabla_{\vec k}E_{cv}(\vec k)}
% \end{split}
%\label{eq:magno-2}
%\end{equation}

%-------------------The Table Of The Paper----------------------
%\begin{table}[!h]
%\tabcolsep 0pt \vspace*{-12pt}
%\caption{The representative $\vec k$ points contributing to $\sigma_2^{xy}$ of interband transition in EuTe around 2.5 eV.}
%\label{Table-EuTe_Sigma}
%\begin{minipage}{\textwidth}
%%\begin{center}
%\centering
%\def\temptablewidth{1.01\textwidth}
%\rule{\temptablewidth}{1pt}
%\begin{tabular*} {\temptablewidth}{@{\extracolsep{\fill}}cccccc}

%-------------------------------------------------------------------------------------------------------------------------
%&Peak (eV)  & {$\vec k$}-point            &Band{$_v$} to Band{$_c$}  &Transition Orbital
%Components\footnote{波函数主要成分后的括号中,$5s$、$5p$和$5p$、$4f$、$5d$分别指碲和铕的原子轨道。} &Gap (eV)   \\ \hline
%-------------------------------------------------------------------------------------------------------------------------
%&2.35       &(0,0,0)         &33$\rightarrow$34    &$4f$(31.58)$5p$(38.69)$\rightarrow$$5p$      &2.142   \\% \cline{3-7}
%&       &(0,0,0)         &33$\rightarrow$34    &$4f$(31.58)$5p$(38.69)$\rightarrow$$5p$      &2.142   \\% \cline{3-7}
%-------------------------------------------------------------------------------------------------------------------------

%\end{tabular*}
%\rule{\temptablewidth}{1pt}\\
%%\end{center}
%\end{minipage}
%\end{table}

%-------------------The Long Table Of The Paper--------------------
%\begin{small}
%%\begin{minipage}{\textwidth}
%%\begin{longtable}[l]{|c|c|cc|c|c|} %[c]指定长表格对齐方式
%\begin{longtable}[c]{|c|c|p{1.9cm}p{4.6cm}|c|c|}
%\caption{Assignment for the peaks of EuB$_6$}
%\label{tab:EuB6-1}\\ %\\长表格的caption中换行不可少
%\hline
%%
%--------------------------------------------------------------------------------------------------------------------------------
%\multicolumn{2}{|c|}{\bfseries$\sigma_1(\omega)$谱峰}&\multicolumn{4}{c|}{\bfseries部分重要能带间电子跃迁\footnotemark}\\ \hline
%\endfirsthead
%--------------------------------------------------------------------------------------------------------------------------------
%%
%\multicolumn{6}{r}{\it 续表}\\
%\hline
%--------------------------------------------------------------------------------------------------------------------------------
%标记 &峰位(eV) &\multicolumn{2}{c|}{有关电子跃迁} &gap(eV)  &\multicolumn{1}{c|}{经验指认} \\ \hline
%\endhead
%--------------------------------------------------------------------------------------------------------------------------------
%%
%\multicolumn{6}{r}{\it 续下页}\\
%\endfoot
%\hline
%--------------------------------------------------------------------------------------------------------------------------------
%%
%%\hlinewd{0.5$p$t}
%\endlastfoot
%--------------------------------------------------------------------------------------------------------------------------------
%%
%% Stuff from here to \endlastfoot goes at bottom of last page.
%%
%--------------------------------------------------------------------------------------------------------------------------------
%标记 &峰位(eV)\footnotetext{见正文说明。} &\multicolumn{2}{c|}{有关电子跃迁\footnotemark} &gap(eV) &\multicolumn{1}{c|}{经验指认\upcite{PRB46-12196_1992}}\\ \hline
%--------------------------------------------------------------------------------------------------------------------------------
%
%     &0.07 &\multicolumn{2}{c|}{电子群体激发$\uparrow$} &--- &电子群\\ \cline{2-5}
%\raisebox{2.3ex}[0pt]{$\omega_f$} &0.1 &\multicolumn{2}{c|}{电子群体激发$\downarrow$} &--- &体激发\\ \hline
%--------------------------------------------------------------------------------------------------------------------------------
%
%     &1.50 &\raisebox{-2ex}[0pt][0pt]{20-22(0,1,4)} &2$p$(10.4)4$f$(74.9)$\rightarrow$ &\raisebox{-2ex}[0pt][0pt]{1.47} &\\%\cline{3-5}
%     &1.50$^\ast$ & &2$p$(17.5)5$d_{\mathrm E}$(14.0)$\uparrow$ & &4$f$$\rightarrow$5$d_{\mathrm E}$\\ \cline{3-5}
%     \raisebox{2.3ex}[0pt][0pt]{$a$} &(1.0$^\dagger$) &\raisebox{-2ex}[0pt][0pt]{20-22(1,2,6)} &\raisebox{-2ex}[0pt][0pt]{4$f$(89.9)$\rightarrow$2$p$(18.7)5$d_{\mathrm E}$(13.9)$\uparrow$}\footnotetext{波函数主要成分后的括号中,2$s$、2$p$和5$p$、4$f$、5$d$、6$s$分别指硼和铕的原子轨道;5$d_{\mathrm E}$、5$d_{\mathrm T}$分别指铕的(5$d_{z^2}$,5$d_{x^2-y^2}$和5$d_{xy}$,5$d_{xz}$,5$d_{yz}$)轨道,5$d_{\mathrm{ET}}$(或5$d_{\mathrm{TE}}$)则指5个5$d$轨道成分都有,成分大的用脚标的第一个字母标示;2$ps$(或2$sp$)表示同时含有硼2$s$、2$p$轨道成分,成分大的用第一个字母标示。$\uparrow$和$\downarrow$分别标示$\alpha$和$\beta$自旋电子跃迁。} &\raisebox{-2ex}[0pt][0pt]{1.56} &激子跃迁。 \\%\cline{3-5}
%     &(1.3$^\dagger$) & & & &\\ \hline
%--------------------------------------------------------------------------------------------------------------------------------

%     & &\raisebox{-2ex}[0pt][0pt]{19-22(0,0,1)} &2$p$(37.6)5$d_{\mathrm T}$(4.5)4$f$(6.7)$\rightarrow$ & & \\\nopagebreak %\cline{3-5}
%     & & &2$p$(24.2)5$d_{\mathrm E}$(10.8)4$f$(5.1)$\uparrow$ &\raisebox{2ex}[0pt][0pt]{2.78} &a、b、c峰可能 \\ \cline{3-5}
%     & &\raisebox{-2ex}[0pt][0pt]{20-29(0,1,1)} &2$p$(35.7)5$d_{\mathrm T}$(4.8)4$f$(10.0)$\rightarrow$ & &包含有复杂的\\ \nopagebreak%\cline{3-5}
%     &2.90 & &2$p$(23.2)5$d_{\mathrm E}$(13.2)4$f$(3.8)$\uparrow$ &\raisebox{2ex}[0pt][0pt]{2.92} &强激子峰。$^{\ast\ast}$\\ \cline{3-5}
%$b$  &2.90$^\ast$ &\raisebox{-2ex}[0pt][0pt]{19-22(0,1,1)} &2$p$(33.9)4$f$(15.5)$\rightarrow$ & &B2$s$-2$p$的价带 \\ \nopagebreak%\cline{3-5}
%     &3.0 & &2$p$(23.2)5$d_{\mathrm E}$(13.2)4$f$(4.8)$\uparrow$ &\raisebox{2ex}[0pt][0pt]{2.94} &顶$\rightarrow$B2$s$-2$p$导\\ \cline{3-5}
%     & &12-15(0,1,2) &2$p$(39.3)$\rightarrow$2$p$(25.2)5$d_{\mathrm E}$(8.6)$\downarrow$ &2.83 &带底跃迁。\\ \cline{3-5}
%     & &14-15(1,1,1) &2$p$(42.5)$\rightarrow$2$p$(29.1)5$d_{\mathrm E}$(7.0)$\downarrow$ &2.96 & \\\cline{3-5}
%     & &13-15(0,1,1) &2$p$(40.4)$\rightarrow$2$p$(28.9)5$d_{\mathrm E}$(6.6)$\downarrow$ &2.98 & \\ \hline
%--------------------------------------------------------------------------------------------------------------------------------
%%\hline
%%\hlinewd{0.5$p$t}
%\end{longtable}
%%\end{minipage}{\textwidth}
%%\setlength{\unitlength}{1cm}
%%\begin{picture}(0.5,2.0)
%%  \put(-0.02,1.93){$^{1)}$}
%%  \put(-0.02,1.43){$^{2)}$}
%%\put(0.25,1.0){\parbox[h]{14.2cm}{\small{\\}}
%%\put(-0.25,2.3){\line(1,0){15}}
%%\end{picture}
%\end{small}

%-----------------------------------------------------------------------------------------------------------------------------------------------------------------------------------------------------%


%--------------------------------------------------------------------------The Biblography of The Paper-----------------------------------------------------------------%
%\newpage																				%
%-----------------------------------------------------------------------------------------------------------------------------------------------------------------------%
%\begin{thebibliography}{99}																		%
%%\bibitem{PRL58-65_1987}H.Feil, C. Haas, {\it Phys. Rev. Lett.} {\bf 58}, 65 (1987).											%
%\end{thebibliography}																			%
%-----------------------------------------------------------------------------------------------------------------------------------------------------------------------%
%																					%
\phantomsection\addcontentsline{toc}{section}{Bibliography}	 %直接调用\addcontentsline命令可能导致超链指向不准确,一般需要在之前调用一次\phantomsection命令加以修正	%
\bibliography{ref/Myref}																			%
\bibliographystyle{ref/mybib}																		%
%  \nocite{*}																				%
%-----------------------------------------------------------------------------------------------------------------------------------------------------------------------%

\clearpage     %\end{CJK} 前加上\clearpage是CJK的要求
\end{document}
