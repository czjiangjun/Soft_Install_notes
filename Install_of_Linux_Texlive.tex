\documentclass[10pt, oneside, a4paper]{article}      % Specifies the document class
%\documentclass[10pt, twoside, a4paper]{article}      % Specifies the document class

%%%%%%%%%%%%%%%%% CJK 中文版面控制  %%%%%%%%%%%%%%%%%%%%%%%%%%%%%%
%\usepackage{CJK} % CTEX-CJK 中文支持                            %
\usepackage{xeCJK} % seperate the english and chinese		 %
\usepackage{CJKutf8} % Texlive 中文支持                         %
\usepackage{CJKnumb} %中文序号                                   %
\usepackage{indentfirst} % 中文段落首行缩进                      %
%\setlength\parindent{22pt}       % 段落起始缩进量               %
\renewcommand{\baselinestretch}{1.2} % 中文行间距调整            %
\setlength{\textwidth}{16cm}                                     %
\setlength{\textheight}{24cm}                                    %
\setlength{\topmargin}{-1cm}                                     %
\setlength{\oddsidemargin}{0.1cm}                                %
\setlength{\evensidemargin}{\oddsidemargin}                      %
\usepackage{fancyhdr}           %使用页眉-页脚                   %
%%%%%%%%%%%%%%%%%%%%%%%%%%%%%%%%%%%%%%%%%%%%%%%%%%%%%%%%%%%%%%%%%%

\usepackage{authblk}					 %作者地址和E-mail
\usepackage{amsmath,amsthm,amsfonts,amssymb,bm}          %数学公式
\usepackage{mathrsfs}                                    %英文花体
\usepackage{xcolor}                                        %使用默认允许使用颜色
%\usepackage{hyperref} 
\usepackage{graphicx}
\usepackage{subfigure}           %图片跨页
\usepackage{animate}		 %插入动画
\usepackage{caption}
\captionsetup{font=footnotesize}

%\usepackage[version=3]{mhchem}		%化学公式
%\usepackage{chemformula}
\usepackage{chemfig}		%化学公式

\usepackage{fontspec} % use to set font
\setCJKmainfont{SimSun}
\XeTeXlinebreaklocale "zh"  % Auto linebreak for chinese
\XeTeXlinebreakskip = 0pt plus 1pt % Auto linebreak for chinese

\usepackage{longtable}                                   %使用长表格
\usepackage{multirow}
\usepackage{makecell}		%允许单元格内换行

\usepackage{arydshln}
\newcommand{\adots}{\mathinner{\mkern2mu%
\raisebox{0.1em}{.}\mkern2mu\raisebox{0.4em}{.}%
\mkern2mu\raisebox{0.7em}{.}\mkern1mu}}
%%%%%%%%%%%%%%%%%%%%%%%%%  参考文献引用 %%%%%%%%%%%%%%%%%%%%%%%%%%%
%%尽量使用 BibTeX(含有超链接,数据库的条目URL即可)                %
%%%%%%%%%%%%%%%%%%%%%%%%%%%%%%%%%%%%%%%%%%%%%%%%%%%%%%%%%%%%%%%%%%%

\usepackage[numbers,sort&compress]{natbib} %紧密排列             %
\usepackage[sectionbib]{chapterbib}        %每章节单独参考文献   %
%\usepackage{footbib}			   %脚注列出参考文献    %
\usepackage{hypernat}                                                                         %
%\usepackage[dvipdfm,bookmarksopen=true,pdfstartview=FitH,CJKbookmarks]{hyperref}              %
\usepackage[bookmarksopen=true,pdfstartview=FitH,CJKbookmarks]{hyperref}              %
\hypersetup{bookmarksnumbered,colorlinks,linkcolor=green,citecolor=blue,urlcolor=red}         %
%参考文献含有超链接引用时需要下列宏包,注意与natbib有冲突        %
%\usepackage[dvipdfm]{hyperref}                                  %
%\usepackage{hypernat}                                           %
\newcommand{\upcite}[1]{\hspace{0ex}\textsuperscript{\cite{#1}}} %

%%%%%%%%%%%%%%%%%%%%%%%%%%%%%%%%%%%%%%%%%%%%%%%%%%%%%%%%%%%%%%%%%%%%%%%%%%%%%%%%%%%%%%%%%%%%%%%
%\AtBeginDvi{\special{pdf:tounicode GBK-EUC-UCS2}} %CTEX用dvipdfmx的话,用该命令可以解决      %
%						   %pdf书签的中文乱码问题		      %
%%%%%%%%%%%%%%%%%%%%%%%%%%%%%%%%%%%%%%%%%%%%%%%%%%%%%%%%%%%%%%%%%%%%%%%%%%%%%%%%%%%%%%%%%%%%%%%

%%%%%%%%%%%%%%%%%%%%%  % 插图使用位置  %%%%%%%%%%%%%%%%%%%%%%%%%%%
\graphicspath{{../Latex_Beamer/}}                            %
%%%%%%%%%%%%%%%%%%%%%%%%%%%%%%%%%%%%%%%%%%%%%%%%%%%%%%%%%%%%%%%%%%

%%%%%%%%%%%%%%%%%%%%%%%%%%%%% 用 authblk 包 支持作者和E-mail %%%%%%%%%%%%%%%%%%%%%%%%%%%%%%%%%
%\title{More than one Author with different Affiliations}				     %
\title{Texlive的安装(xelatex中文字体和手工宏包加载)}   %
%\author[a]{Author A}									     %
\author[]{ }   %
%\author[a]{Author B}									     %
%\author[a]{Author C \thanks{Corresponding author: email@mail.com}}			     %
%\author[a]{Author/通讯作者 C \thanks{Corresponding author: cores-email@mail.com}}     %
%\author[b]{Author D}									     %
%\author[b]{Author/作者 D}									     %
%\author[b]{Author E}									     %
%\affil[a]{Department of Computer Science, \LaTeX\ University}				     %
%\affil[a]{作者单位-1 \authorcr 地址}    %\authorcr表示换行
%\affil[b]{Department of Mechanical Engineering, \LaTeX\ University}			     %
%\affil[b]{作者单位-2}			     %
											     %
%%% 使用 \thanks 定义通讯作者								     %
%%\affil命令后的{}中的内容,如果觉得需要换行的话,换行命令是\authorcr(不是\\)。
%%Email中可以吧相同邮箱的人@前面的内容写在一个{}里,用逗号隔开。注意{和}前面要加\。例如:
%%\affil[*]{单位1, \authorcr Email: \{zuozhe1, zuozhe2\}@yahoo.com, zuozhe3@sina.com}
											     %
\renewcommand*{\Authfont}{\small\rm} % 修改作者的字体与大小				     %
\renewcommand*{\Affilfont}{\small\it} % 修改机构名称的字体与大小			     %
%\renewcommand\Authands{ and } % 去掉 and 前的逗号					     %
\renewcommand\Authands{ , } % 将 and 换成逗号					     %
\date{} % 去掉日期									     %
%%%%%%%%%%%%%%%%%%%%%%%%%%%%%%%%%%%%%%%%%%%%%%%%%%%%%%%%%%%%%%%%%%%%%%%%%%%%%%%%%%%%%%%%%%%%%%

\begin{document}
%%%%%%%%%%%%%%%%%%%%%  % 页眉-页脚设计  %%%%%%%%%%%%%%%%%%%%%%%%%%%
%\pagestyle{fancy}    %与文献引用超链接style有冲突
%\lhead{\bfseries Result} %页眉左边位置内容,并加粗 
%\chead{} % 页眉中间位置内容
\rhead{\includegraphics[scale=0.20]{Figures/BCC_logo-1.png}}%在此处插入logo.pdf图片 图片靠右
%\lfoot{}  %页脚
%\cfoot{}
%\rfoot{}
%%%%%%%%%%%%%%%%%  % pagestyleR常用格式  %%%%%%%%%%%%%%%%%%%%%%%%%
%% empty 无页眉页脚
%% plain 无页眉,页脚为居中页码
%% headings 页眉为章节标题,无页脚
%% myheadings 页眉内容可自定义,无页脚
%%%%%%%%%%%%%%%%%%%%%%%%%%%%%%%%%%%%%%%%%%%%%%%%%%%%%%%%%%%%%%%%%%

%\begin{CJK}{UTF8}{gbsn} %针对文字编码为unix %CJK自带的utf-8简体字体有gbsn(宋体)和gkai(楷体)
%\begin{CJK}{GBK}{hei}	%针对文字编码为doc
%\begin{CJK}{GBK}{hei}	 %针对文字编码为doc
%\CJKindent     %在CJK环境中,中文段落起始缩进2个中文字符
%\indent
%
\renewcommand{\abstractname}{\small{\CJKfamily{hei} 摘\quad 要}} %\CJKfamily{hei} 设置中文字体,字号用\big \small来设
\renewcommand{\refname}{\centering\CJKfamily{hei} 参考文献}
%\renewcommand{\figurename}{\CJKfamily{hei} 图.}
\renewcommand{\figurename}{{\bf Fig}.}
%\renewcommand{\tablename}{\CJKfamily{hei} 表.}
\renewcommand{\tablename}{{\bf Tab}.}
%\renewcommand{\thesubfigure}{\roman{subfigure}}  \makeatletter %子图标记罗马字母
%\renewcommand{\thesubfigure}{\tiny(\alph{subfigure})}  \makeatletter %子图标记英文字母
%\renewcommand{\thesubfigure}{}  \makeatletter %子图无标记

%将图表的Caption写成 图(表) Num. 格式
\makeatletter
\long\def\@makecaption#1#2{%
  \vskip\abovecaptionskip
  \sbox\@tempboxa{#1. #2}%
  \ifdim \wd\@tempboxa >\hsize
    #1. #2\par
  \else
    \global \@minipagefalse
    \hb@xt@\hsize{\hfil\box\@tempboxa\hfil}%
  \fi
  \vskip\belowcaptionskip}
\makeatother

\newcommand{\keywords}[1]{{\hspace{0\ccwd}\small{\CJKfamily{hei} 关键词:}{\hspace{2ex}{#1}}\bigskip}}

%%%%%%%%%%%%%%%%%%中文字体设置%%%%%%%%%%%%%%%%%%%%%%%%%%%
%默认字体 defalut fonts \TeX 是一种排版工具 \\		%
%{\bfseries 粗体 bold \TeX 是一种排版工具} \\		%
%{\CJKfamily{song}宋体 songti \TeX 是一种排版工具} \\	%
%{\CJKfamily{hei} 黑体 heiti \TeX 是一种排版工具} \\	%
%{\CJKfamily{kai} 楷书 kaishu \TeX 是一种排版工具} \\	%
%{\CJKfamily{fs} 仿宋 fangsong \TeX 是一种排版工具} \\	%
%%%%%%%%%%%%%%%%%%%%%%%%%%%%%%%%%%%%%%%%%%%%%%%%%%%%%%%%%

%\addcontentsline{toc}{section}{Bibliography}

%%%%%%%%%%%%%%%%%%%%%%%%%%%%%%%%%%%%%%%%%%  不使用 authblk 包制作标题  %%%%%%%%%%%%%%%%%%%%%%%%%%%%%%%%%%%%%%%%%%%%%%
%-------------------------------The Title of The Paper--------------------------------------------------------------%
%\title{标题}
%-------------------------------------------------------------------------------------------------------------------%

%----------------------The Authors and the address of The Paper-----------------------------------------------------%
%\author{ %%作者:
%\small
%北京市计算中心~~姜骏\thanks{jiangjun@bcc.ac.cn} %报告式:~单位:~作者
%Author1, Author2, Author3\thanks{Communication author's E-mail} \\    %Authors' Names	                           %
%\small
%(The Address,City Post code)						%Address	                            %
%}
%\affil[$\dagger$]{清华大学~材料加工研究所~A213}                                                                    %
%\affil{清华大学~材料加工研究所~A213}
%\date{}					%if necessary				              	            %
%-------------------------------------------------------------------------------------------------------------------%
%%%%%%%%%%%%%%%%%%%%%%%%%%%%%%%%%%%%%%%%%%%%%%%%%%%%%%%%%%%%%%%%%%%%%%%%%%%%%%%%%%%%%%%%%%%%%%%%%%%%%%%%%%%%%%%%%%%%%
\maketitle
%\thispagestyle{fancy}   % 首页插入页眉页脚 

%-------------------------------------------------------------------------------The Abstract and the keywords of The Paper----------------------------------------------------------------------------%
%\begin{abstract}
%The content of the abstract
%\end{abstract}

%\keywords{Keyword1; Keyword2; Keyword3}

%-------------------------------------------------------------------------------The Content of The Paper----------------------------------------------------------------------------------------------%
%\tableofcontents %% 制作目录(目录是根据标题自动生成的)
%----------------------------------------------------------------------------------------------------------------------------------------------------------------------------------------------------%

%\newpage	        % 每个新的/newpage 即可有新的\thispagestyle 引领      %
%\thispagestyle{fancy}   % 插入页眉页脚                                        %
%----------------------------------------------------------------------------------------The Main Body Of The Paper----------------------------------------------------------------------------------------%
%Introduction

%\setcounter{section}%{-1}% 控制chapter/section/page等的序号
\section{\textrm{Texlive}的安装}
以\textrm{ubuntu18.04}为例
\begin{enumerate}
	\item 访问以下网址下载texlive的ISO文件\\
		\url{https://mirrors.tuna.tsinghua.edu.cn/#}\\
		\textrm{wget~-c~https://mirrors.tuna.tsinghua.edu.cn/CTAN/systems/texlive/Images/texlive2020-20200406.iso}
	\item 安装\textrm{per}组件\\
终端运行以下命令\\
\textrm{sudo~apt-get~install~perl-tk}\\
\textrm{sudo~apt-get~install~texlive-full}\\
\textrm{sudo~apt-get~install~texstudio}\\
\item 加载该ISO文件\\
	\textrm{sudo~mount~-o~loop~texlive2018.iso~/mnt}~~换掉文件路径即可)\\
	注意:使用该命令会出现错误提示,\textrm{mount:~/dev/loop1~is~write-protected,~mounting~read-only.}不必管它
\item 启动图形化安装界面\\
	\textrm{cd~/mnt}\\
	\textrm{sudo~./install-tl~-gui}\\
点击安装即可\\
安装时间比较长耐心等待,安装完成后执行以下操作
\item 为了支持中文,我们需要使用\textrm{usepackage\{xeCJK\}}包,所以需要安装:\\
	\textrm{sudo~apt-get~install~texlive-lang-chinese}
\item 要使用更多软件包和字体进行更完整的安装运行以下命令\\
	\textrm{sudo~apt-get~install~texlive-latex-base~texlive-latex-extra~texlive-latex-recommended~texlive-fonts-recommended}
\item 安装\textrm{XeLatex}\\
	\textrm{sudo~apt-get~install~texlive-xetex}
\item 安装\textrm{texstudio}编辑器
	\textrm{sudo~apt-get~install~texstudio}
\end{enumerate}
至此\textrm{texlive}的安装就完成了,可以正常输出中文进行编译.

\section{Windows字体安装}
\begin{enumerate}
	\item 拷贝中文字体进入 \textrm{/usr/share/fonts/Windows}~里面,没有此文件夹自己创建\\
		常用的6个字体是:~宋体(\textrm{simsun.ttf})、仿宋(\textrm{simfang.ttf})、黑体(\textrm{simhei.ttf})、楷体(\textrm{simkai.ttf})、隶书(\textrm{simli.ttf})、幼圆(\textrm{simyou.ttf})
	\item 执行如下命令:~
\$~\textrm{cd~/usr/share/fonts/Windows}\\
\$~\textrm{chmod~777~*} \#因为原有权限为600,改为644/777\\
\$~\textrm{mkfontscale}\\
\$~\textrm{mkfontdir}\\
\$~\textrm{fc-cache}
\item 测试字体是否安装完毕
	\$~\textrm{fc-list}
\end{enumerate}
如果有上面的字体就是安装成功了

\section{手工添加缺失宏包并生效}

\begin{itemize}
	\item 方法一:~手动添加宏包,在以下任意一个目录下放上\textrm{.sty}文件(宏包对应文件)
		\begin{enumerate}
			\item \textrm{/usr/local/share/texmf/}
			\item \textrm{/var/lib/texmf:/usr/share/texmf/}
			\item \textrm{/usr/share/texlive/texmf-dist/} \qquad\#\textcolor{red}{默认建议用\textrm{/usr/share/texlive/texmf-dist/tex/latex/}}
		\end{enumerate}
然后执行命令\\
\textrm{sudo~texhash} \qquad\# 建议具体到\textrm{.sty}所在目录下用$\ast$标记:~\textrm{sudo~texhash~~/usr/share/texlive/texmf-dist/tex/latex/*}\\
\# \textcolor{red}{默认建议命令为}:~\textrm{sudo~texhash~~/usr/share/texlive/texmf-dist/tex/latex/*}\\
然后理应就找到了
\item 方法二:~进入\textrm{/home}目录,在\textrm{/home}目录下创建如下文件夹\textrm{texmf/tex/latex},假设我们添加的宏包叫做\textrm{foo.sty},就创建名字为\textrm{foo}的文件夹:
	\begin{enumerate}
		\item \textrm{cd~~~\~}
		\item \textrm{mkdir~-p~texmf/tex/latex/foo}
		\item 将\textrm{foo.sty}放入\textrm{foo}文件夹中
	\end{enumerate}
再执行下面的命令\\
\textrm{texhash~\~/texmf/*} \#建议具体到\textrm{.sty}所在文件\\
就没事儿了

\textcolor{red}{命令\textrm{texhash}与\textrm{mktexlsr}效果相同,但区别是}:
\begin{itemize}
	\item \textrm{texhash}~针对的是目录
	\item \textrm{mktexlsr}~针对的是具体文件
\end{itemize}
所以建议两个命令都用!
\end{itemize}

\section{常用的编辑工具}
\begin{itemize}
	\item \textrm{pdf}编辑:~\textrm{pdfarranger}
	\item 图片编辑:~\textrm{gimp}
	\item 摄像头驱动:~\textrm{guvcview}
	\item 在线视频:~\textrm{obs-studio}
	\item \textrm{Origin}:~\textrm{labplot}
	\item \textrm{视频编辑}:~\textrm{kdenlive}
	\item \textrm{cooledit}:~\textrm{audacity}
	\item \textrm{千千静听}:~\textrm{qmmp}
	\item \textrm{FTP}:~\textrm{FileZilla}
	\item \textrm{VPN}:~\textrm{MozillaVPN}
\end{itemize}

\section{Virtualbox}
\textcolor{blue}{\textrm{VirtualBox kernel driver not installed.\\
The vboxdrv kernel module was either not loaded or /dev/vboxdrv was not created for some reason.\\
‘/etc/init.d/vboxdrv setup’\\
as root.\\
VBox status code: -1908 (VERR\_VM\_DRIVER\_NOT\_INSTALLED)}} \\

按照提示在终端里输入\\
\textcolor{red}{\textrm{sudo /etc/init.d/vboxdrv setup}}\\
仍然不行。给的信息是:\\

\textcolor{blue}{\textrm{Stopping VirtualBox kernel modules ...done.\\
Recompiling VirtualBox kernel modules ...failed!\\
  (Look at /var/log/vbox-install.log to find out what went wrong)}} \\

打开/var/log/vbox-install.log文件\\

\textcolor{blue}{\textrm{Makefile:73: *** Error: unable to find the sources of your current Linux kernel. Specify KERN\_DIR= and run Make again.}}\\

更新的时候升级了\textrm{Linux kernel},所以\textrm{vboxdrv}需要重新编译。因为是从源里面直接更新的,没有Linux kernel的源文件,显然编译需要这些源文件,最后解决方法如下\\

\textcolor{red}{\textrm{sudo apt-get install linux-headers-\$(uname -r)\\
	\textrm{sudo apt-get install dkms}\\
	\textrm{sudo /etc/init.d/vboxdrv setup}}}

\section{\rm{Ubuntu}下的有线网络链接}
有线宽带在\textrm{Windows}下可以正常连接,不过在\textrm{Ubuntu}下死活连不上,基本可以排除宽带线路问题和机器的网卡硬件问题,所以问题应该是出在\textrm{Ubuntu}系统设置或驱动上。解决方式:
\begin{enumerate}
	\item 启动\textrm{NetworkManager}:\\
		\textrm{sudo vim /etc/NetworkManager/NetworkManager.conf} \\
		\#修改managed=true
	\item 重启服务\\
		\textrm{sudo service network-manager restart}
	\item \textrm{ifconfig}查看网卡 \\
		\# 查看启用的网卡:~ \textrm{ifconfig}\\
		\# 查看所有网卡:~\textrm{ ifconfig~-a}\\
一般宽带使用的为\textrm{enp}开头的网卡(如:~\textrm{enp7s0}),如果网卡未启用,则通过\\
\textrm{sudo ifconfig enp7s0 up}启用网卡(\textrm{down}为关闭)
	\item 通过\textrm{nmcli}命令查看网卡状态,有没有被\textrm{network-manager}所管理:
\begin{figure}[h!]
\centering
\includegraphics[height=3.35in,clip]{Figures/ubuntu_nmcli.jpg}
%\caption{\small Band Structure of PbTe (a) and EuTe (b).}%(与文献\cite{EPJB33-47_2003}图1对比)
\label{ubuntu_nmcli}
\end{figure}
\end{enumerate}

如果\textrm{enp7s0}显示为未托管状态,则需要修改配置文件\\
\# 备份原配置文件\\
\textrm{sudo mv /usr/lib/NetworkManager/conf.d/10-globally-managed-devices.conf  /usr/lib/NetworkManager/conf.d/10-globally-managed-devices.conf\_origin}\\
\# 新建新配置文件\\
\textrm{sudo touch /usr/lib/NetworkManager/conf.d/10-globally-managed-devices.conf}\\
修改后重启\textrm{networkmaneger}服务:\\
\textrm{sudo service network-manager restart}\\
再用\textrm{nmcli}命令查看,会发现\textrm{enp7s0}已经被\textrm{network-manager}所管理,在系统$\rightarrow$ 设置$\rightarrow$ 网络中发现,多出来了“有线”选项,表示可以联网了
\begin{figure}[h!]
\centering
\includegraphics[height=2.35in,clip]{Figures/ubuntu_network.jpg}
%\caption{\small Band Structure of PbTe (a) and EuTe (b).}%(与文献\cite{EPJB33-47_2003}图1对比)
\label{ubuntu_network}
\end{figure}

%-------------------The Figure Of The Paper------------------
%\begin{figure}[h!]
%\centering
%\includegraphics[height=3.35in,width=2.85in,viewport=0 0 400 475,clip]{PbTe_Band_SO.eps}
%\hspace{0.5in}
%\includegraphics[height=3.35in,width=2.85in,viewport=0 0 400 475,clip]{EuTe_Band_SO.eps}
%\caption{\small Band Structure of PbTe (a) and EuTe (b).}%(与文献\cite{EPJB33-47_2003}图1对比)
%\label{Pb:EuTe-Band_struct}
%\end{figure}

%-------------------The Equation Of The Paper-----------------
%\begin{equation}
%\varepsilon_1(\omega)=1+\frac2{\pi}\mathscr P\int_0^{+\infty}\frac{\omega'\varepsilon_2(\omega')}{\omega'^2-\omega^2}d\omega'
%\label{eq:magno-1}
%\end{equation}

%\begin{equation} 
%\begin{split}
%\varepsilon_2(\omega)&=\frac{e^2}{2\pi m^2\omega^2}\sum_{c,v}\int_{BZ}d{\vec k}\left|\vec e\cdot\vec M_{cv}(\vec k)\right|^2\delta [E_{cv}(\vec k)-\hbar\omega] \\
% &= \frac{e^2}{2\pi m^2\omega^2}\sum_{c,v}\int_{E_{cv}(\vec k=\hbar\omega)}\left|\vec e\cdot\vec M_{cv}(\vec k)\right|^2\dfrac{dS}{\nabla_{\vec k}E_{cv}(\vec k)}
% \end{split}
%\label{eq:magno-2}
%\end{equation}

%-------------------The Table Of The Paper----------------------
%\begin{table}[!h]
%\tabcolsep 0pt \vspace*{-12pt}
%%\caption{The representative $\vec k$ points contributing to $\sigma_2^{xy}$ of interband transition in EuTe around 2.5 eV.}
%\label{Table-EuTe_Sigma}
%\begin{minipage}{\textwidth}
%%\begin{center}
%\centering
%\def\temptablewidth{0.84\textwidth}
%\rule{\temptablewidth}{1pt}
%\begin{tabular*} {\temptablewidth}{|@{\extracolsep{\fill}}c|@{\extracolsep{\fill}}c|@{\extracolsep{\fill}}l|}

%-------------------------------------------------------------------------------------------------------------------------
%&Peak (eV)  & {$\vec k$}-point            &Band{$_v$} to Band{$_c$}  &Transition Orbital
%Components\footnote{波函数主要成分后的括号中,$5s$、$5p$和$5p$、$4f$、$5d$分别指碲和铕的原子轨道。} &Gap (eV)   \\ \hline
%-------------------------------------------------------------------------------------------------------------------------
%&2.35       &(0,0,0)         &33$\rightarrow$34    &$4f$(31.58)$5p$(38.69)$\rightarrow$$5p$      &2.142   \\% \cline{3-7}
%&       &(0,0,0)         &33$\rightarrow$34    &$4f$(31.58)$5p$(38.69)$\rightarrow$$5p$      &2.142   \\% \cline{3-7}
%-------------------------------------------------------------------------------------------------------------------------
%\end{tabular*}
%\rule{\temptablewidth}{1pt}
%\end{minipage}{\textwidth}
%\end{table}

%-------------------The Long Table Of The Paper--------------------
%\begin{small}
%%\begin{minipage}{\textwidth}
%%\begin{longtable}[l]{|c|c|cc|c|c|} %[c]指定长表格对齐方式
%\begin{longtable}[c]{|c|c|p{1.9cm}p{4.6cm}|c|c|}
%\caption{Assignment for the peaks of EuB$_6$}
%\label{tab:EuB6-1}\\ %\\长表格的caption中换行不可少
%\hline
%%
%--------------------------------------------------------------------------------------------------------------------------------
%\multicolumn{2}{|c|}{\bfseries$\sigma_1(\omega)$谱峰}&\multicolumn{4}{c|}{\bfseries部分重要能带间电子跃迁\footnotemark}\\ \hline
%\endfirsthead
%--------------------------------------------------------------------------------------------------------------------------------
%%
%\multicolumn{6}{r}{\it 续表}\\
%\hline
%--------------------------------------------------------------------------------------------------------------------------------
%标记 &峰位(eV) &\multicolumn{2}{c|}{有关电子跃迁} &gap(eV)  &\multicolumn{1}{c|}{经验指认} \\ \hline
%\endhead
%--------------------------------------------------------------------------------------------------------------------------------
%%
%\multicolumn{6}{r}{\it 续下页}\\
%\endfoot
%\hline
%--------------------------------------------------------------------------------------------------------------------------------
%%
%%\hlinewd{0.5$p$t}
%\endlastfoot
%--------------------------------------------------------------------------------------------------------------------------------
%%
%% Stuff from here to \endlastfoot goes at bottom of last page.
%%
%--------------------------------------------------------------------------------------------------------------------------------
%标记 &峰位(eV)\footnotetext{见正文说明。} &\multicolumn{2}{c|}{有关电子跃迁\footnotemark} &gap(eV) &\multicolumn{1}{c|}{经验指认\upcite{PRB46-12196_1992}}\\ \hline
%--------------------------------------------------------------------------------------------------------------------------------
%
%     &0.07 &\multicolumn{2}{c|}{电子群体激发$\uparrow$} &--- &电子群\\ \cline{2-5}
%\raisebox{2.3ex}[0pt]{$\omega_f$} &0.1 &\multicolumn{2}{c|}{电子群体激发$\downarrow$} &--- &体激发\\ \hline
%--------------------------------------------------------------------------------------------------------------------------------
%
%     &1.50 &\raisebox{-2ex}[0pt][0pt]{20-22(0,1,4)} &2$p$(10.4)4$f$(74.9)$\rightarrow$ &\raisebox{-2ex}[0pt][0pt]{1.47} &\\%\cline{3-5}
%     &1.50$^\ast$ & &2$p$(17.5)5$d_{\mathrm E}$(14.0)$\uparrow$ & &4$f$$\rightarrow$5$d_{\mathrm E}$\\ \cline{3-5}
%     \raisebox{2.3ex}[0pt][0pt]{$a$} &(1.0$^\dagger$) &\raisebox{-2ex}[0pt][0pt]{20-22(1,2,6)} &\raisebox{-2ex}[0pt][0pt]{4$f$(89.9)$\rightarrow$2$p$(18.7)5$d_{\mathrm E}$(13.9)$\uparrow$}\footnotetext{波函数主要成分后的括号中,2$s$、2$p$和5$p$、4$f$、5$d$、6$s$分别指硼和铕的原子轨道;5$d_{\mathrm E}$、5$d_{\mathrm T}$分别指铕的(5$d_{z^2}$,5$d_{x^2-y^2}$和5$d_{xy}$,5$d_{xz}$,5$d_{yz}$)轨道,5$d_{\mathrm{ET}}$(或5$d_{\mathrm{TE}}$)则指5个5$d$轨道成分都有,成分大的用脚标的第一个字母标示;2$ps$(或2$sp$)表示同时含有硼2$s$、2$p$轨道成分,成分大的用第一个字母标示。$\uparrow$和$\downarrow$分别标示$\alpha$和$\beta$自旋电子跃迁。} &\raisebox{-2ex}[0pt][0pt]{1.56} &激子跃迁。 \\%\cline{3-5}
%     &(1.3$^\dagger$) & & & &\\ \hline
%--------------------------------------------------------------------------------------------------------------------------------

%     & &\raisebox{-2ex}[0pt][0pt]{19-22(0,0,1)} &2$p$(37.6)5$d_{\mathrm T}$(4.5)4$f$(6.7)$\rightarrow$ & & \\\nopagebreak %\cline{3-5}
%     & & &2$p$(24.2)5$d_{\mathrm E}$(10.8)4$f$(5.1)$\uparrow$ &\raisebox{2ex}[0pt][0pt]{2.78} &a、b、c峰可能 \\ \cline{3-5}
%     & &\raisebox{-2ex}[0pt][0pt]{20-29(0,1,1)} &2$p$(35.7)5$d_{\mathrm T}$(4.8)4$f$(10.0)$\rightarrow$ & &包含有复杂的\\ \nopagebreak%\cline{3-5}
%     &2.90 & &2$p$(23.2)5$d_{\mathrm E}$(13.2)4$f$(3.8)$\uparrow$ &\raisebox{2ex}[0pt][0pt]{2.92} &强激子峰。$^{\ast\ast}$\\ \cline{3-5}
%$b$  &2.90$^\ast$ &\raisebox{-2ex}[0pt][0pt]{19-22(0,1,1)} &2$p$(33.9)4$f$(15.5)$\rightarrow$ & &B2$s$-2$p$的价带 \\ \nopagebreak%\cline{3-5}
%     &3.0 & &2$p$(23.2)5$d_{\mathrm E}$(13.2)4$f$(4.8)$\uparrow$ &\raisebox{2ex}[0pt][0pt]{2.94} &顶$\rightarrow$B2$s$-2$p$导\\ \cline{3-5}
%     & &12-15(0,1,2) &2$p$(39.3)$\rightarrow$2$p$(25.2)5$d_{\mathrm E}$(8.6)$\downarrow$ &2.83 &带底跃迁。\\ \cline{3-5}
%     & &14-15(1,1,1) &2$p$(42.5)$\rightarrow$2$p$(29.1)5$d_{\mathrm E}$(7.0)$\downarrow$ &2.96 & \\\cline{3-5}
%     & &13-15(0,1,1) &2$p$(40.4)$\rightarrow$2$p$(28.9)5$d_{\mathrm E}$(6.6)$\downarrow$ &2.98 & \\ \hline
%--------------------------------------------------------------------------------------------------------------------------------
%%\hline
%%\hlinewd{0.5$p$t}
%\end{longtable}
%%\end{minipage}{\textwidth}
%%\setlength{\unitlength}{1cm}
%%\begin{picture}(0.5,2.0)
%%  \put(-0.02,1.93){$^{1)}$}
%%  \put(-0.02,1.43){$^{2)}$}
%%\put(0.25,1.0){\parbox[h]{14.2cm}{\small{\\}}
%%\put(-0.25,2.3){\line(1,0){15}}
%%\end{picture}
%\end{small}

%-----------------------------------------------------------------------------------------------------------------------------------------------------------------------------------------------------%

%-------------------------------------------------------------------------Thanks------------------------------------------------------------------------------------------------
%\newpage %%
%\newpage %%
%\thispagestyle{fancy}   % 首页插入页眉页脚 
%-----------------------------------------------------------------------------------------------------------------------------------------------------------------------%

%--------------------------------------------------------------------------The Biblography of The Paper-----------------------------------------------------------------%
%\newpage																				%
%-----------------------------------------------------------------------------------------------------------------------------------------------------------------------%
%\begin{thebibliography}{99}																		%
%%\bibitem{PRL58-65_1987}H.Feil, C. Haas, {\it Phys. Rev. Lett.} {\bf 58}, 65 (1987).											%
%	\bibitem{kp-method} \textrm{Zhenxi Pan, Yong Pan, Jun Jiang$^{\ast}$, Liutao Zhao}, \textrm{High-Throughput Electronic Band Structure Calculations for Hexaborides}, \textit{Intelligent Computing}, \textbf{Springer}, \textbf{P.386-395}, (2019).%
%	\bibitem{PAW-dataset} \textrm{姜骏},\textrm{PAW原子数据集的构造与检验}, \textit{中国化学会第十二届全国量子化学会议论文摘要集},\textbf{太原},(2014).
%\end{thebibliography}																			%
%-----------------------------------------------------------------------------------------------------------------------------------------------------------------------%
%\phantomsection\addcontentsline{toc}{section}{Bibliography} %直接调用\addcontentsline命令可能导致超链指向不准确,一般需要在之前调用一次\phantomsection命令加以修正%
%%\bibliography{../ref/Myref_HT}   %
%\bibliography{../ref/Myref_from_2013}   %
%\bibliographystyle{../ref/mybib} %% 接近ieeert样式
%%%%%%%%%%%%%%%%%%%%%%%%%%%%      \bibliographystyle         %%%%%%%%%%%%%%%%%%%%%%%%%%%%%%%%%%
%%%%%%      LaTeX 参考文献标准选项及其样式共有以下8种:                                %%%%%%%%
% plain,按字母的顺序排列,比较次序为作者、年度和标题.
% unsrt,样式同plain,只是按照引用的先后排序.
% alpha,用作者名首字母+年份后两位作标号,以字母顺序排序.
% abbrv,类似plain,将月份全拼改为缩写,更显紧凑.
% ieeetr,国际电气电子工程师协会期刊样式.
% acm,美国计算机学会期刊样式.
% siam,美国工业和应用数学学会期刊样式.
% apalike,美国心理学学会期刊样式.
%%%%%%%%%%%%%%%%%%%%%%%%%%%%%%%%%%%%%%%%%%%%%%%%%%%%%%%%%%%%%%%%%%%%%%%%%%%%%%%%%%%%%%%%%%%%%%%
%  \nocite{*}																				%
%-----------------------------------------------------------------------------------------------------------------------------------------------------------------------%

\clearpage     %\end{CJK} 前加上\clearpage是CJK的要求
%\end{CJK*}
\end{document}
